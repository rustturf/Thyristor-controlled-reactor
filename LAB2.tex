\documentclass[a4paper,12pt]{report}

\usepackage[utf8]{inputenc}
\usepackage{graphicx}
\usepackage{geometry}
\geometry{left=1in,right=1in,top=1in,bottom=1in}
\usepackage{setspace}
\usepackage{longtable}
\usepackage{array}
\usepackage{booktabs}
\usepackage{caption}
\usepackage{multirow}
\usepackage{adjustbox}
\usepackage{rotating}
\usepackage{pdfpages}
\usepackage{amsmath}
\usepackage{xspace}
\usepackage{textcomp}
\usepackage{lscape}
\usepackage{amssymb}
\usepackage{bigstrut}
\usepackage{tcolorbox}

\setlength{\parindent}{0pt}
\renewcommand\thesection{\arabic{section}}

\begin{document}

\begin{center}
    \includegraphics[width=4cm]{Figures/KhCE.png} \\
    \vspace{0.5cm}
    \textbf{(AN UNDERTAKING OF BHAKTAPUR MUNICIPALITY)}\\
    \vspace{0.5cm}
    \textbf{\LARGE KHWOPA COLLEGE OF ENGINEERING}\\
    \textbf{AFFILIATED TO TRIBHUVAN UNIVERSITY}
\end{center}

\vspace{1.5cm}
\begin{center}
    \textbf{\Large A}\\
    \textbf{\Large Report on}\\
    \vspace{0.5cm}
    \textbf{\large LAB 2: THYRISTOR CONTROLLED REACTOR}\\[0.3cm]
\end{center}

\vspace{0.5cm}
\begin{center}
    \textbf{Submitted By : }
\end{center}

\begin{center}
    Shubhanga Aryal \qquad KCE078BEL038
\end{center}

\vspace*{1cm}
\begin{center}
    \textbf{Submitted To: }\\[0.2cm]
    \textbf{Er. Sabin Kasula}\\[0.2cm]
    \textbf{Lecturer, KhCE}
\end{center}

\vspace{0.3cm}
\begin{center}
\textbf{\Large Department of Electrical Engineering}\\
\vspace{0.3cm}
\textbf{KHWOPA COLLEGE OF ENGINEERING}\\
\vspace{0.3cm}
\textbf{Libali-08, Bhaktapur}
\end{center}

\thispagestyle{empty}
\newpage

\onehalfspacing

\begin{center}
    \Large LAB 2\\[0.2cm]
    \Large THYRISTOR CONTROLLED REACTOR
\end{center}

\section{Objectives}
To understand the working of Thyristor Controlled Reactor and simulate it using Matlab/Simulink for various firing angles.

\section{Software Used}
\begin{itemize}
    \item Matlab/Simulink
\end{itemize}

\section{Theory}
A Thyristor Controlled Reactor (TCR) is a reactance connected in series with a bidirectional thyristor valve.
The thyristor valve is phase controlled, which allows the value of delivered reactive power to be varied to meet different system conditions.
TCRs are used in transmission lines for limiting voltages on lightly loaded transmission lines.\\[0.2cm]

The circuit diagram of a thyristor controlled reactor is as shown below:

\begin{figure}[!hb]
    \centering
    \includegraphics[width=0.5\textwidth]{Figures/figure1.png}
    \caption{Circuit Diagram of Thyristor Controlled Reactor}
    \label{fig:TCR}
\end{figure}

\subsection*{Operating Principle of TCR}
A TCR scheme has a fixed inductor $L$ in series with an AC voltage controller consisting of thyristors $T_1$ and $T_2$ as shown in the figure above.
The thyristor $T_1$ conducts for the positive half cycle of the reactor current and thyristor $T_2$ conducts for the negative half cycle of the reactor current.\\[0.2cm]

The basic idea is to change the RMS value of the reactor current $I_L$ by controlling the firing angle $\alpha$.
The magnitude of the reactor current changes with constant system voltage $V_s$ and the inductor in series with the voltage controller is equivalent to $X_L(\alpha)$ as a function of firing angle.\\[0.2cm]

Applying KVL to the circuit, we have:

\begin{equation}
    V_s = L\dfrac{di_L}{dt} + R i_L
\end{equation}

The solution of the above differential equation is given by a steady and transient part as:

\begin{equation}
i_L = \dfrac{V_m}{Z}\sin(\omega t - \phi) + A e^{-\frac{R}{L}t}
\end{equation}

where 
\[
Z = \sqrt{R^2 + (\omega L)^2}
\]
and
\[
\phi = \tan^{-1}\!\left(\dfrac{\omega L}{R}\right)
\]

Solving the above equations with given initial conditions and simplifying, we get:

\begin{equation}
i_L(t) =
\dfrac{V_m}{Z}
\left[
\sin(\omega t - \phi)
-
\sin(\alpha - \phi)
e^{-\frac{R}{L}\left(t-\frac{\alpha}{\omega}\right)}
\right]
\end{equation}
If $\alpha=\phi$, the current becomes continuous and sinusoidal.\\[0.2cm]
Since $X_L>>>>R$ the above equation becomes:
\begin{equation}
i_L(t) = \dfrac{V_m}{Z}sin(\omega t - \frac{\pi}{2})-\dfrac{V_m}{Z}sin(\alpha - \frac{\pi}{2})
\end{equation}
When alpha is equal to 90 degrees, the second part of the current becomes zero .
This means that the reactor current lags the source voltage by 90 degrees. Hence alpha=90 degree is taken as the 
firing angle $\delta=0$ to analyze the circuit. 

\[
\text{When } \alpha = \frac{\pi}{2} + \delta \quad (\text{i.e., firing angle } = \delta)
\]

\[
i_L = \frac{V_m}{Z} \sin\left(\alpha t - \frac{\pi}{2}\right)
      - \frac{V_m}{Z} \sin \delta
\]

From point \(a\) to \(b\) in Figure 2,

\[
\frac{V_m}{Z} \sin\left(\alpha t - \frac{\pi}{2}\right)
<
\frac{V_m}{Z} \sin \delta
\]
\begin{figure}
    \centering
    \includegraphics[width=0.7\textwidth]{Figures/figure2.png}
    \caption{Waveform of TCR for Firing Angle $\delta$}
    \label{fig:TCR_waveform} 
\end{figure}
Therefore, from point a to b net current iL is negative, hence T1
does not conduct from a to b. Therefore, the current waveform is not sindusodial.\\[0.2cm]
Upon fourier analysis and simplyfying the fundamental component of inductor current is:
\begin{equation}
    I_l=\dfrac{V_m}{X_L}(1- \dfrac{2\delta}{\pi}-\dfrac{sin2\delta}{\pi}).
\end{equation}
The above equation can be written as : 
\begin{equation}
    I_L=B(\delta)V_m
\end{equation}
where $B(\delta)=1/X=\dfrac{1}{\omega L}(1- \dfrac{2\delta}{\pi}-\dfrac{sin2\delta}{\pi})$.
and $X_L=\omega L(\dfrac{\pi}{\pi-2\delta+sin2\delta})$.\\[0.2cm]
From the above equation it is clear that the inductive reactance $X_L$ is a function of firing angle $\delta$.\\[0.2cm]
Since, inductor draws non sinusodial current, it causes harmonics in the system.
\newpage
\section{Observation}
The circuit diagram to simulate the thyristor controlled reactor is as shown below: 
\begin{figure}[!hb]
    \centering
    \includegraphics[width=0.9\textwidth,height=0.5\textwidth]{Figures/figure3.png}
    \caption{Circuit Diagram of TCR For Simulation}
    \label{fig:ckt diagram} 
\end{figure}
The thyristor was fired for different for different firing angles as shown:\\[0.2cm]
For $\alpha=90^\circ$:
\begin{figure}[!hb]
    \centering
    \includegraphics[width=1\textwidth]{Figures/figure4.png}
    \caption{Output Waveforms for $\alpha=90^\circ$}
    \label{fig: 90 degree}
\end{figure}\\
\newpage
For $\alpha=110^\circ$:
\begin{figure}[!hb]
    \centering
    \includegraphics[width=1\textwidth]{Figures/figure6.png}
    \caption{Output Waveforms for $\alpha=110^\circ$}
    \label{fig:110 degree}
\end{figure}\\
For $\alpha=170^\circ$:
\begin{figure}[!hb]
    \centering
    \includegraphics[width=1\textwidth]{Figures/figure8.png}
    \caption{Output Waveforms for $\alpha=170^\circ$}
    \label{fig:170 degree}
\end{figure}\\
\newpage
For $\alpha=180^\circ$:
\begin{figure}[!hb]
    \centering
    \includegraphics[width=1\textwidth]{Figures/figure10.png}
    \caption{Output Waveforms for $\alpha=180^\circ$}
    \label{fig: 180 degree}
\end{figure}
\section{Discussion}
The simulation results show the expected Operating characteristics of a thyristor controlled reactor as the firing angle alpha is varied at $\alpha=90^\circ$. The thyristors conduct for the entire half cycle producing a nearly sinusoidal inductor current with maximum magnitude which indicates full reactive power absorption. The inductor voltage closely follows the source during conduction while the thyristor voltage remains close to zero except during the blocking intervals. The gate pulses coincide with the voltage zero crossing conforming continuous conduction.\\[0.2cm]
When the firing angle is increased to 110 degrees, conduction becomes partial and the inductor current magnitude decreases noticeably. The current waveform becomes more distorted, and sharp changes appear in the inductor voltage as the firing and extinction instants due to sudden variation in current in the inductor current . The voltage across the thyristor  alternate between near zero values during conduction and the source voltage during blocking, while the delayed gate signals clearly demonstrate the phase control action.\\[0.2cm]
At $\alpha = 170^\circ$, the thyristors conduct only for very short intervals, causing the reactor current to appear as small pulses with extremely low peak value, corresponding to almost zero reactive power absorption. The inductor voltage is significant only during these brief conduction periods, and the thyristor voltage follows the source for most of the cycle, indicating prolonged blocking. Overall, increasing $\alpha$ reduces the effective reactor current and increases waveform distortion, validating the controllable nature of the TCR for reactive-power regulation.\\[0.2cm]
At $\alpha = 180^\circ$, the thyristors are commanded to fire at the negative peak of the source voltage, effectively blocking all conduction throughout the cycle. The reactor current is essentially zero, indicating that no reactive power is delivered to the system. The inductor voltage remains negligible as there is no significant current flow, while the thyristor voltage tracks the entire source voltage waveform, demonstrating complete blocking. The gate pulses are shifted by the maximum delay, and the system behaves as an open circuit with infinite equivalent reactance. This represents the extreme case where the TCR is completely disabled and provides minimum reactive power support.
\section{Conclusion}
Hence, the working and effect of various firing angles on thyristor controlled reactor was studied and analyzed.
\end{document}
